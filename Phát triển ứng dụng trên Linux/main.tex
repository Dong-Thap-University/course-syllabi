\documentclass[a4paper,13pt]{article}

\usepackage[utf8]{inputenc}
\usepackage[vietnamese]{babel}
\usepackage{geometry}
\usepackage{longtable}
\usepackage{enumitem}
\usepackage{setspace}
\usepackage{array}
\usepackage{multirow}
\usepackage{tabularx}
\usepackage{ltablex}
\usepackage{amssymb}
\usepackage{amsmath}
\newcommand{\checkedbox}{{\small$\ooalign{$\square$\cr\hfil{\scriptsize$\checkmark$}\hfil\cr}$}}
\newcommand{\uncheckedbox}{{\small$\square$}}
\keepXColumns
\usepackage{booktabs}
\newcolumntype{Y}{>{\raggedright\arraybackslash}X}
\linespread{1.5}
\geometry{margin=2.5cm}
\setlength{\parskip}{0.5em}
\setlength{\parindent}{0pt}

% Đảm bảo tiêu đề các mục dùng cỡ chữ 13pt giống thân bài
\usepackage{sectsty}
\sectionfont{\fontsize{13pt}{13pt}\selectfont}
\subsectionfont{\fontsize{13pt}{13pt}\selectfont}


\begin{document}

% ===== Tiêu ngữ đầu trang =====
\begin{center}
\begin{tabular}{p{0.5\textwidth} p{0.55\textwidth}}
\centering
\begin{tabular}{c}
\textbf{BỘ GIÁO DỤC VÀ ĐÀO TẠO} \\
\textbf{TRƯỜNG ĐẠI HỌC ĐỒNG THÁP} \\
\rule{5cm}{0.4pt}
\end{tabular}
&
\makebox[0.55\textwidth][c]{%
  \begin{tabular}{c}
    \textbf{CỘNG HÒA XÃ HỘI CHỦ NGHĨA VIỆT NAM} \\
    \textbf{Độc lập – Tự do – Hạnh phúc} \\
    \rule{6.5cm}{0.4pt}
  \end{tabular}
}
\end{tabular}
\end{center}

% ===== Tiêu đề ĐỀ CƯƠNG CHI TIẾT HỌC PHẦN =====
\begin{center}
\textbf{\fontsize{13pt}{13pt}\selectfont ĐỀ CƯƠNG CHI TIẾT HỌC PHẦN}
\end{center}


\subsection*{1. Thông tin chung về học phần}
\begin{tabular}{|c|p{5cm}|p{9cm}|}
\hline
\textbf{STT} & \textbf{TÊN} & \textbf{NỘI DUNG} \\
\hline
1 & Tên học phần (tiếng Việt): & Phát triển ứng dụng trên Linux \\
\hline
2 & Tên học phần (tiếng Anh): & Application Development on Linux \\
\hline
3 & Mã học phần: & \\
\hline
4 & Số tín chỉ: & 2 \\
\hline
5 & Phân bổ thời gian: & \\
\hline
  & - Lý thuyết: & 20 \\
\hline
  & - Thực hành/Thí nghiệm: & 20 \\
\hline
  & - Tự học: & 60 \\
\hline
6 & Bộ môn quản lý: & Khoa Công nghệ và Kỹ thuật \\
\hline
7 & Điều kiện tham gia học phần: & \\
\hline
  & - Học phần tiên quyết: & \\
\hline
  & - Học phần học trước: & \\
\hline
  & - Học phần song hành: & \\
\hline
8 & Loại học phần: & \checkedbox{} Bắt buộc \quad \uncheckedbox{} Tự chọn \\
\hline
9 & Trình độ: & Đại học \\
\hline
10 & Thuộc khối kiến thức: & \\
\hline
\  & & \uncheckedbox{} Kiến thức giáo dục đại cương \\
\hline
\  & & \uncheckedbox{} Kiến thức cơ sở nhóm ngành \\
\hline
\  & & \uncheckedbox{} Kiến thức cơ sở ngành \\
\hline
\  & & \checkedbox{} Kiến thức chuyên ngành \\
\hline
\  & & \uncheckedbox{} Kiến thức NVSP (đối với ngành sư phạm) \\
\hline
\  & & \uncheckedbox{} Kiến thức bổ trợ (đối với ngành ngoài sư phạm) \\
\hline
\  & & \uncheckedbox{} Kiến thức thực tập nghề nghiệp (đối với ngành ngoài SP) \\
\hline
\  & & \uncheckedbox{} Khóa luận tốt nghiệp/Đồ án tốt nghiệp/Học phần thay thế \\
\hline
\end{tabular}

\subsection*{2. Giảng viên phụ trách}
\subsubsection*{2.1. Giảng viên 1}
\begin{itemize}
    \item Họ và tên: Lê Minh Thư
    \item E-mail: 
    \item Điện thoại: 
\end{itemize}

\subsubsection*{2.2. Giảng viên 2}
\begin{itemize}
    \item Họ và tên: Nguyễn Tuấn Kiệt
    \item E-mail: 
    \item Điện thoại:
\end{itemize}

\subsection*{3. Mô tả tóm tắt học phần}
\hspace{1cm}Học phần "Phát triển ứng dụng trên Linux" cung cấp cho sinh viên nền tảng kiến thức cơ bản về hệ điều hành Linux và môi trường phát triển ứng dụng. Sinh viên được trang bị kiến thức về kiến trúc hệ điều hành, cách thức quản lý tệp tin, người dùng và quyền truy cập, cùng với các lệnh cơ bản để làm việc hiệu quả trên nền tảng Linux.

\hspace{1cm}Nội dung học phần tập trung vào lập trình hệ thống với các chủ đề quan trọng như xử lý tệp tin (file), quản lý tiến trình (process), xử lý tín hiệu (signal), lập trình đa luồng (thread), và các kỹ thuật giao tiếp liên tiến trình (IPC) bao gồm pipe, socket và shared memory. Sinh viên cũng được học cách sử dụng các công cụ phát triển chuyên nghiệp như GCC, GDB, Make, và CMake.

\hspace{1cm}Cuối cùng, học phần chú trọng vào việc ứng dụng thực tế thông qua việc xây dựng các ứng dụng dòng lệnh và ứng dụng giao diện đồ họa đơn giản sử dụng GTK+. Sinh viên được rèn luyện kỹ năng làm việc nhóm, quản lý dự án và triển khai các sản phẩm phần mềm hoàn chỉnh trên môi trường Linux.

\subsection*{4. Mục tiêu học phần}
\begin{enumerate}[label=4.\arabic*.]
    \item Trình bày được kiến trúc và đặc trưng của hệ điều hành Linux, cũng như các công cụ phát triển cơ bản.
    \item Vận dụng thành thạo kỹ năng lập trình hệ thống trong Linux như xử lý tiến trình, tín hiệu, và giao tiếp liên tiến trình.
    \item Xây dựng được ứng dụng thực tế trên nền tảng Linux bằng dòng lệnh hoặc giao diện đồ hoạ.
    \item Phát triển kỹ năng tự học, tư duy hệ thống và làm việc nhóm thông qua thực hành và dự án.
\end{enumerate}

\subsection*{4.1. Kiến thức}
\begin{enumerate}[label=4.1.\arabic*.]
    \item Trình bày được kiến trúc hệ điều hành Linux, cơ chế quản lý người dùng, hệ thống tệp tin và phân quyền truy cập.
    \item Mô tả và sử dụng được các công cụ phát triển phần mềm phổ biến trên Linux: Emacs, Gedit, GNU Make, Automake, Autoconf.
    \item Trình bày nguyên lý và viết được chương trình shell script với các cấu trúc điều khiển cơ bản.
    \item Hiểu và áp dụng được công cụ Qt Creator để xây dựng ứng dụng giao diện đồ họa, kết nối cơ sở dữ liệu MySQL/SQL Server, xử lý mạng và lập trình song song.
\end{enumerate}

\subsection*{4.2. Kỹ năng}
\begin{enumerate}[label=4.2.\arabic*.]
    \item Khai thác được hệ điều hành Linux để thao tác với tệp tin, người dùng, phân quyền và thực thi lệnh shell.
    \item Cài đặt và sử dụng thành thạo các công cụ hỗ trợ phát triển phần mềm trên Linux (GCC, Gedit, Makefile, Autotools…).
    \item Viết được shell script để xử lý dữ liệu và tự động hóa quy trình; triển khai ứng dụng kết nối cơ sở dữ liệu và lập trình song song trên Qt Creator.
    \item Vận dụng tổng hợp kiến thức và công cụ để phát triển ứng dụng thực tế trên môi trường Linux.
\end{enumerate}

\subsection*{5. Chuẩn đầu ra của học phần (Course Learning Outcomes: CLOs)}
\begin{longtable}{|>{\centering\arraybackslash}m{2cm}|>{\raggedright\arraybackslash}m{13cm}|}
\hline
\textbf{CĐR (CLOs)} & \multicolumn{1}{|>{\centering\arraybackslash}m{13cm}|}{\textbf{Nội dung CĐR học phần (CLOs)}} \\
\hline
\textbf{CLO1:} & Hiểu các nguyên lý của hệ điều hành Linux và đồng lực chuyển từ phân tích dữ liệu truyền thống sang phân tích dữ liệu quy mô lớn. \\
\hline
\textbf{CLO2:} & Giải thích các yếu tố cơ bản trong quản lý dữ liệu quy mô lớn: khả năng mở rộng, xử lý song song trên hệ thống lưu trữ phân tán, và xử lý dữ liệu luồng. \\
\hline
\textbf{CLO3:} & Thao tác với Apache Hadoop, Apache Spark và MongoDB như các công cụ nhập môn cho phân tích dữ liệu quy mô lớn. \\
\hline
\textbf{CLO4:} & Đạt được cái nhìn tổng quan về Big Data trong thực tiễn thông qua việc nghiên cứu và chia sẻ kinh nghiệm về các ứng dụng, công cụ và kỹ thuật. \\
\hline
\textbf{CLO5:} & Tham gia một khóa học trực tuyến được chỉ định về một số chủ đề Big Data và hoàn thành hầu hết các nội dung học tập trong khóa học đó. \\
\hline
\textbf{CLO6:} & Hiểu các nguyên lý của Big Data và đồng lực chuyển từ phân tích dữ liệu truyền thống sang phân tích dữ liệu quy mô lớn. \\
\hline
\end{longtable}

\subsection*{6. Ma trận thể hiện sự đóng góp của các chuẩn đầu ra học phần (CLO) vào việc đạt được các chuẩn đầu ra của CTĐT (PLO) và các chỉ số PI (Performance Indicator)}

\centerline{\textbf{Bảng 6.1. Mối liên hệ giữa CLO và PLO/PI}}

\begin{center}
\begin{longtable}{|>{\centering\arraybackslash}m{2cm}|>{\centering\arraybackslash}m{3cm}|>{\centering\arraybackslash}m{3cm}|>{\centering\arraybackslash}m{3cm}|>{\centering\arraybackslash}m{3cm}|}
\hline
\multirow{3}{*}{\textbf{CLO}} & \multicolumn{4}{c|}{\textbf{PLO và chỉ số PI}} \\
\cline{2-5}
& \multicolumn{2}{c|}{\textbf{PLO4}} & \multicolumn{2}{c|}{\textbf{PLO5}} \\
\cline{2-5}
& \textbf{PI4.1} & \textbf{PI4.2} & \textbf{PI5.1} & \textbf{PI5.2} \\
\hline
\textbf{CLO1} & & & & \\
\hline
\textbf{CLO2} & & & & \\
\hline
\textbf{CLO3} & & & & \\
\hline
\textbf{CLO4} & & & & \\
\hline
\textbf{CLO5} & & & & \\
\hline
\textbf{CLO6} & & & & \\
\hline
\textbf{Học phần} & & & & \\
\hline
\end{longtable}
\end{center}

\subsection*{7. Nội dung học phần và kế hoạch dạy học}

\subsubsection*{7.1. Lý thuyết}
\begin{longtable}{|>{\raggedright\arraybackslash}p{6cm}|>{\centering\arraybackslash}p{1.5cm}|p{4.5cm}|>{\centering\arraybackslash}p{2.5cm}|>{\centering\arraybackslash}p{1.5cm}|}
\hline
\textbf{Chương/chủ đề} & \textbf{Số tiết} & \multicolumn{2}{c|}{\textbf{Hoạt động dạy và học}} & \textbf{CĐR học phần có liên quan} \\
\cline{3-4}
& & \textbf{Phương pháp dạy học} & \textbf{Sinh viên chuẩn bị} & \\
\hline
\textbf{CHƯƠNG 1. Linux và Công nghệ Container}\newline
1.1 Tổng quan về Linux\newline
1.2 Các lệnh và quản lý hệ thống Linux cơ bản\newline
1.3 Tổng quan về công nghệ Container\newline
1.4 Docker và Podman cơ bản\newline
1.5 Quản lý container và image & 5 & GV thuyết trình, minh họa; SV trao đổi & Xem tài liệu Docker & CLO1 \\
\hline
\textbf{CHƯƠNG 2. Công cụ phát triển hiện đại}\newline
2.1 Giới thiệu VS Code\newline
2.2 Quản lý mã nguồn với Git\newline
2.3 Phát triển ứng dụng với Docker\newline
2.4 Tự động hóa phát triển với GitHub Actions & 5 & GV thuyết trình, minh họa; SV thực hành & Xem GitHub Actions Documentation & CLO2 \\
\hline
\textbf{CHƯƠNG 3. Lập trình hệ thống và phát triển container}\newline
3.1 Các khái niệm cơ bản về lập trình hệ thống\newline
3.2 Lập trình quản lý tiến trình và file\newline
3.3 Kết nối mạng và IPC\newline
3.4 Phát triển ứng dụng container hóa\newline
3.5 Docker Compose cơ bản & 5 & GV thuyết trình, thực hành mẫu & Xem Docker Compose Documentation & CLO3 \\
\hline
\textbf{CHƯƠNG 4. Microservices và Kubernetes}\newline
4.1 Kiến trúc Microservices\newline
4.2 Thiết kế và xây dựng REST APIs\newline
4.3 Kubernetes cơ bản\newline
4.4 Triển khai ứng dụng với Kubernetes\newline
4.5 CI/CD với Kubernetes và GitHub Actions\newline
4.6 Giám sát và quản lý với Prometheus & 5 & GV hướng dẫn, SV thực hành nhóm & Xem Kubernetes Documentation & CLO4 \\
\hline
\end{longtable}

\subsubsection*{7.2. Thực hành}
\begin{longtable}{|>{\raggedright\arraybackslash}p{0.38\textwidth}|>{\centering\arraybackslash}p{0.07\textwidth}|>{\raggedright\arraybackslash}p{0.22\textwidth}|>{\raggedright\arraybackslash}p{0.15\textwidth}|>{\centering\arraybackslash}p{0.11\textwidth}|}
\hline
\multicolumn{1}{|>{\centering\arraybackslash}p{0.38\textwidth}|}{\textbf{Chương/chủ đề}} & \multicolumn{1}{|>{\centering\arraybackslash}p{0.07\textwidth}|}{\textbf{Số tiết}} & \multicolumn{2}{c|}{\textbf{Hoạt động dạy và học}} & \multicolumn{1}{|>{\centering\arraybackslash}p{0.11\textwidth}|}{\textbf{Hình thức đánh giá}} \\
\cline{3-4}
\multicolumn{1}{|>{\centering\arraybackslash}p{0.38\textwidth}|}{} & \multicolumn{1}{|>{\centering\arraybackslash}p{0.07\textwidth}|}{} & \multicolumn{1}{|>{\centering\arraybackslash}p{0.22\textwidth}|}{\textbf{Phương pháp dạy học}} & \multicolumn{1}{|>{\centering\arraybackslash}p{0.15\textwidth}|}{\textbf{Sinh viên chuẩn bị}} & \multicolumn{1}{|>{\centering\arraybackslash}p{0.11\textwidth}|}{} \\
\hline
\endfirsthead
\hline
\multicolumn{1}{|>{\centering\arraybackslash}p{0.38\textwidth}|}{\textbf{Chương/chủ đề}} & \multicolumn{1}{|>{\centering\arraybackslash}p{0.07\textwidth}|}{\textbf{Số tiết}} & \multicolumn{2}{c|}{\textbf{Hoạt động dạy và học}} & \multicolumn{1}{|>{\centering\arraybackslash}p{0.11\textwidth}|}{\textbf{Hình thức đánh giá}} \\
\cline{3-4}
\multicolumn{1}{|>{\centering\arraybackslash}p{0.38\textwidth}|}{} & \multicolumn{1}{|>{\centering\arraybackslash}p{0.07\textwidth}|}{} & \multicolumn{1}{|>{\centering\arraybackslash}p{0.22\textwidth}|}{\textbf{Phương pháp dạy học}} & \multicolumn{1}{|>{\centering\arraybackslash}p{0.15\textwidth}|}{\textbf{Sinh viên chuẩn bị}} & \multicolumn{1}{|>{\centering\arraybackslash}p{0.11\textwidth}|}{} \\
\hline
\endhead
\textbf{CHƯƠNG 1:} \newline Cài đặt Ubuntu/CentOS trên máy ảo hoặc WSL \newline Thực hành các lệnh Linux cơ bản (ls, cd, mkdir, chmod, grep, find) \newline Cài đặt và cấu hình Docker/Podman \newline Tạo và quản lý container đầu tiên \newline Làm việc với Docker images và volumes & 5 & Thực hành cá nhân & Chuẩn bị máy ảo hoặc WSL, tài khoản Docker Hub & Bài tập thực hành \\
\hline
\textbf{CHƯƠNG 2:} \newline Cài đặt và cấu hình VS Code với các extension Linux \newline Thực hành Git: clone, commit, push, pull, branch, merge \newline Tạo Dockerfile và build image \newline Thiết lập GitHub repository và Actions \newline Tạo CI/CD pipeline đơn giản với GitHub Actions & 5 & Thực hành nhóm & Tạo tài khoản GitHub, đọc Git documentation & Báo cáo nhóm \\
\hline
\textbf{CHƯƠNG 3:} \newline Lập trình xử lý file và process trong C/Python \newline Thực hành IPC: pipe, socket, shared memory \newline Viết shell script tự động hóa tác vụ \newline Tạo multi-container application với Docker Compose \newline Kết nối database (MySQL/PostgreSQL) trong container & 5 & Thực hành cá nhân/nhóm & Chuẩn bị mã nguồn C/Python, học Docker Compose & Bài tập thực hành \\
\hline
\textbf{CHƯƠNG 4:} \newline Cài đặt và cấu hình Kubernetes cluster (minikube/kind) \newline Tạo Deployment, Service, ConfigMap, Secret \newline Triển khai ứng dụng web với database trên K8s \newline Thiết lập monitoring với Prometheus và Grafana \newline Tạo complete CI/CD pipeline từ code đến production & 5 & Thực hành nhóm, trình bày & Đọc Kubernetes documentation, chuẩn bị demo & Báo cáo nhóm \\
\hline
\end{longtable}

\subsection*{8. Tài liệu học tập}
\begin{longtable}{|>{\centering\arraybackslash}p{0.8cm}|>{\centering\arraybackslash}p{2.5cm}|>{\raggedright\arraybackslash}p{3.5cm}|>{\centering\arraybackslash}p{1.2cm}|>{\centering\arraybackslash}p{2cm}|>{\centering\arraybackslash}p{1.5cm}|>{\centering\arraybackslash}p{2.5cm}|}
\hline
\multirow{2}{*}{\textbf{TT}} & \multirow{2}{*}{\textbf{Tên tác giả}} & \multirow{2}{*}{\textbf{Tên tài liệu}} & \multirow{2}{*}{\textbf{Năm xuất bản}} & \multirow{2}{*}{\textbf{Nhà xuất bản}} & \multirow{2}{*}{\textbf{Địa chỉ khai thác tài liệu}} & \textbf{Mục đích sử dụng} \\
\cline{7-7}
& & & & & & \textbf{Tài liệu chính} \quad \textbf{Tham khảo} \\
\hline
\endfirsthead
\hline
\multirow{2}{*}{\textbf{TT}} & \multirow{2}{*}{\textbf{Tên tác giả}} & \multirow{2}{*}{\textbf{Tên tài liệu}} & \multirow{2}{*}{\textbf{Năm xuất bản}} & \multirow{2}{*}{\textbf{Nhà xuất bản}} & \multirow{2}{*}{\textbf{Địa chỉ khai thác tài liệu}} & \textbf{Mục đích sử dụng} \\
\cline{7-7}
& & & & & & \textbf{Tài liệu chính} \quad \textbf{Tham khảo} \\
\hline
\endhead
1 & Nigel Poulton & Docker Deep Dive & 2023 & Independently published & Thư viện & x \\
\hline
2 & Brendan Burns, Joe Beda & Kubernetes: Up and Running & 2022 & O'Reilly Media & Thư viện & x \\
\hline
3 & Docker Inc. & Docker Documentation & 2024 & Docker Inc. & https://docs.docker.com & x \\
\hline
4 & CNCF & Kubernetes Documentation & 2024 & CNCF & https://kubernetes.io/docs & x \\
\hline
5 & GitHub & GitHub Actions Documentation & 2024 & GitHub Inc. & https://docs.github.com/actions & x \\
\hline
6 & Prometheus & Prometheus Documentation & 2024 & Prometheus & https://prometheus.io/docs
\hline
\end{longtable}

\subsection*{9. Quy định đối với sinh viên}
\begin{itemize}
    \item Tuân thủ nghiêm túc các nội quy và quy định của Khoa và Trường khi vào lớp học.
    \item Không được vắng quá 20\% số tiết học.
    \item Mọi hành vi gian lận trong học tập bị xử lý và cho 0 điểm.
    \item Tài khoản trực tuyến phải định danh bằng MSSV, họ tên và ảnh thật.
\end{itemize}

\subsection*{10. Phương pháp – hình thức đánh giá kết quả học tập}
\begin{longtable}{|>{\centering\arraybackslash}p{1cm}|p{3cm}|p{5.5cm}|>{\centering\arraybackslash}p{2cm}|>{\centering\arraybackslash}p{2cm}|}
\hline
\textbf{STT} & \textbf{Hình thức đánh giá} & \textbf{Nội dung} & \textbf{CĐR liên kết} & \textbf{Trọng số} \\
\hline
\endfirsthead
\hline
\textbf{STT} & \textbf{Hình thức đánh giá} & \textbf{Nội dung} & \textbf{CĐR liên kết} & \textbf{Trọng số} \\
\hline
\endhead
1 & Chuyên cần & Đi học đúng giờ, thực hành tích cực & 5.3.1, 5.3.2 & 0.1 \\
\hline
2 & Trắc nghiệm giữa kỳ & Câu hỏi lý thuyết chương 1, 2 & 5.1.1, 5.1.2 & 0.2 \\
\hline
3 & Bài tập thực hành & File, signal, process, IPC & 5.2.1, 5.1.3 & 0.2 \\
\hline
4 & Coding Test & Lập trình hệ thống và debug & 5.1.2, 5.2.1 & 0.2 \\
\hline
5 & Dự án cuối kỳ & Project CLI/GUI, teamwork & 5.2.2, 5.3.2 & 0.3 \\
\hline
\end{longtable}

\vspace{2em}
\begin{flushright}
    Đồng Tháp, ngày \rule{1cm}{0.2pt} tháng \rule{1cm}{0.2pt} năm \rule{1.5cm}{0.2pt}
\end{flushright}

\begin{center}
    \begin{tabular}{>{\centering\arraybackslash}p{5cm} >{\centering\arraybackslash}p{5cm} >{\centering\arraybackslash}p{5cm}}
        \textbf{Trưởng khoa} & \textbf{Trưởng bộ môn} & \textbf{Giảng viên} \\
        (Ký tên, ghi rõ họ tên) & (Ký tên, ghi rõ họ tên) & (Ký tên, ghi rõ họ tên kiet) \\
    \end{tabular}
\end{center}

\end{document}