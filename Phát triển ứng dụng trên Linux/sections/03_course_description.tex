\subsection*{3. Mô tả tóm tắt học phần}
\hspace{1cm}Học phần “Phát triển ứng dụng trên Linux” hướng đến việc trang bị cho sinh viên các kiến thức và kỹ năng thiết yếu để phát triển phần mềm thực tế trên môi trường hệ điều hành Linux. Sinh viên sẽ được làm quen với quy trình phát triển một ứng dụng hoàn chỉnh, từ thiết kế backend, giao diện người dùng, quản lý cơ sở dữ liệu đến triển khai hệ thống.

\hspace{1cm}Cụ thể, sinh viên sẽ học cách xây dựng backend RESTful API bằng FastAPI, phát triển giao diện ứng dụng bằng PyQt5, và thao tác với cơ sở dữ liệu PostgreSQL. Quá trình học sẽ tích hợp sử dụng Git/GitHub để quản lý mã nguồn, viết tài liệu kỹ thuật bằng Markdown, và triển khai toàn bộ hệ thống bằng Docker.

\hspace{1cm}Học phần được thiết kế theo hướng thực hành, giúp sinh viên rèn luyện khả năng làm việc nhóm, tư duy hệ thống và kỹ năng phát triển phần mềm toàn diện. Cuối học phần, sinh viên có thể phát triển, tài liệu hóa và triển khai một ứng dụng đầy đủ chức năng, có thể chạy trên môi trường Linux hiện đại.
