\subsection*{10. Phương pháp, hình thức kiểm tra – đánh giá của học phần}
Kết quả học tập của sinh viên được đánh giá bằng các thành phần: đánh giá quá trình, đánh giá giữa kỳ, đánh giá cuối kỳ, các hoạt động đánh giá khác.

\begin{tabularx}{\textwidth}{|
>{\centering\arraybackslash}X|
>{\centering\arraybackslash}X|
>{\raggedright\arraybackslash}X|
>{\raggedright\arraybackslash}X|
>{\centering\arraybackslash}X|
>{\centering\arraybackslash}X|
>{\centering\arraybackslash}X|
>{\centering\arraybackslash}X|}
\hline
\textbf{Thành phần đánh giá} &
\textbf{Hình thức đánh giá} &
\textbf{Phương pháp đánh giá} &
\textbf{Tiêu chí đánh giá} &
\textbf{Trọng số thành phần (\%)} &
\textbf{Trọng số CLO trong thành phần đánh giá (\%)} &
\textbf{Trọng số đánh giá (\%)} &
\textbf{Lấy dữ liệu đo lường mức độ đạt PLO/PI} \\
\hline
\textbf{HĐ1} & Chuyên cần & Kiểm tra thường xuyên & Rubric (1) & 10\% & CLO6 & 100\% & \\
\hline
\textbf{HĐ2} & Kiểm tra & Làm bài trắc nghiệm & Rubric (2) & 20\% & CLO1, CLO2 & 50\% & \\
\hline
\textbf{HĐ3} & Kiểm tra & Bài tập thực hành & Rubric (3) & 20\% & CLO3, CLO4, CLO5 & 50\% & \\
\hline
\textbf{HĐ4} & Báo cáo cuối môn & Trình bày theo nhóm & Rubric (4) & 60\% & CLO1, CLO6 & 100\% & PI4.1 \\
\hline
\end{tabularx}